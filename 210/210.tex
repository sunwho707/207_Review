\documentclass{article}
\usepackage[a4paper,top=1cm,bottom=2cm,left=1cm,right=1cm]{geometry}
\usepackage{amsmath}
\begin{document}
\Large
  \section{Formulas}
  \textbf{Maxwell's Equations}
  $$ \nabla .D = \rho $$
  $$\nabla .B = 0$$
  $$\nabla \times E = -\frac{\delta B}{\delta t}$$
  $$\nabla \times H = J_C + \frac{\delta D}{\delta t}$$
  
  \textbf{Electric Field}
  $$E = \frac{Q}{4\pi \epsilon r^2}$$, where $\epsilon = \epsilon_0 \epsilon_r$, $\epsilon_0 = 8.85 \times 10^{-12}Fm^{-1}$
  
  $$E = -grad(V)$$
  
  \textbf{Electric Flux}
  $$\Psi = \iint\epsilon E ds = \iint Dds $$
  \textbf{Electric Flux Density}
  $$D = \frac{\Psi}{A}$$

    
  
  \textbf{Magnetic Flux}
  $$ \Phi = \iint \mu Hds = \iint Bds $$
  
  \textbf{Magnetic Flux Density}
  $$B = \frac{\Phi}{A} = \mu H$$
  
  \section{Definitions}
  \begin{itemize}
  \item \textbf{Gauss's Law}: Total electric flux over a volumn is equal to the charge enclosed by that volumn.
  \item \textbf{Electric Field}:
  \item \textbf{Absolute Potential}: The work move a unit charge from infinity to a radial distance r1. 
  \item \textbf{Electric Flux}: Electric Flux through a surface is the integral of normal component of electric field multiplied by $\epsilon$.
  \item \textbf{Electric Flux Density}: Electric flux divided by A.
  \item \textbf{Permittivity}:Permittivity of vacuum multiplied by relative permittivity.
  \item \textbf{Drift Velocity}:
  \item \textbf{Magnatic Flux Density}:
  \item \textbf{Relative Permeability}:
  \item \textbf{Transmission Line}:
  \item \textbf{Application of Transmission Lines}:
  \item \textbf{VSWR}:
  \item \textbf{AC Circuit Theory}:
  \end{itemize}
  
  
  
  \section{Tao Lu}
\end{document}